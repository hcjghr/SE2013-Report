% Chapter 8

\chapter{C++  vs MATLAB} % Main chapter title

\label{Chapter8} % For referencing the chapter elsewhere, use \ref{Chapter1} 

\lhead{Chapter 8. \emph{C++  vs MATLAB}} % This is for the header on each page - perhaps a shortened title

%----------------------------------------------------------------------------------------

\paragraph{}
	First of all the main difference between C++ and MATLAB is its cost. There are a lot of free and commercial implementations of C++ for different platforms. MATLAB on its term is only in charge. Of course student version of MATLAB exists, but it does not have some of toolboxes that can be very useful. 

\paragraph{}
	Beside this as we repeatedly noticed MATLAB has extremely high CPU and memory usage in comparison with C++. So it takes a lot of time for huge data calculation.

\paragraph{}
	From other point of view MATLAB is more flexible tool. Even if it takes more resources for data calculation representation of the data is more “free”. It is much easier to work with matrices, strings and other data in MATLAB than in C++. There is no need in variables declaration in MATLAB like in C++, we can use any variable in any part of program without caring of its type.

\paragraph{}
	Another MATLAB particularity is its global variables. In C++ global variable that declared in the beginning of the program became global for all program cycle. In MATLAB declared global variable is global only inside function. To use it in another function we need also declare it as a global in this function.

\paragraph{}
	One more difference between MATLAB and C++ is a huge amount of toolboxes in MATLAB. It allows to make different experiments without big effort. For example plotting graphs, made difficult calculations.

\paragraph{}
	Other difference is graphical user interface creation. C++ provides an extensive choice of methods and tools for GUI implementation. MATLAB in its turn has limited set of tools for this purpose, that does not give such big opportunities like in C++.

\paragraph{}
	Thus we can conclude that both C++ and MATALB are powerful tools, but must take into account that they should be chosen correctly depends on the existing problem.

