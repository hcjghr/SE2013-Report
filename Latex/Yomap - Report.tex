\documentclass[reqno,openany,12pt]{amsbook}
\usepackage{amsmath}

%  openany option dumps the blank pages between chapters when the next
%  chapter starts on odd page number; following command had similar
%  effect
%  \let\cleardoublepage\clearpage
%  NB need an abstract or get a blank page between title and contents


\renewcommand{\baselinestretch}{1.35}


%  ************  begin my definitions  *******************

% *** theorem etc. commands ***

\newtheorem{thm}{Theorem}%[section]
\newtheorem{lemma}[thm]{Lemma}
\newtheorem{cor}[thm]{Corollary}

\theoremstyle{definition}
\newtheorem{definition}[thm]{Definition}
\newtheorem*{ass}{Assumption S}

\theoremstyle{remark}
\newtheorem{remark}[thm]{Remark}

%\numberwithin{equation}{section}

\newenvironment{mylist}{\begin{enumerate}
\def\labelenumi{\theenumi}
\renewcommand{\theenumi}{(\roman{enumi})}
}{\end{enumerate}}

% *** greek commands ***

\newcommand\al{\alpha}
\newcommand\be{\beta}
\newcommand\ga{\gamma}
\newcommand\Ga{\Gamma}
\newcommand\de{\delta}
\newcommand\De{\Delta}
\newcommand\ep{\epsilon}
\newcommand\ka{\kappa}
\newcommand\la{\lambda}
\newcommand\La{\Lambda}
\newcommand\om{\omega}
\newcommand\Om{\Omega}
\newcommand\si{\sigma}
\newcommand\Si{\Sigma}
\renewcommand\th{\theta}
\newcommand\Th{\Theta}

% *** tilde/bar/bold commands ***


\newcommand\bg{\bar g}
\newcommand\bh{\bar h}
\newcommand\bu{\bar u}

\newcommand\bq{\boldsymbol{q}}
\newcommand\br{\boldsymbol{r}}
\newcommand\bv{\boldsymbol{v}}

% *** Bbb commands ***

\newcommand\Q{\mathbb{Q}}
\newcommand\R{\mathbb{R}}
\newcommand\Nf{\mathbb{N}}
\newcommand\Zf{\mathbb{Z}}

% *** script commands ***

\newcommand\I{{\mathcal I}}

\newcommand\B{{\mathcal B}}

% *** brackets commands ***

\newcommand\lan{\langle}
\newcommand\ran{\rangle}

% *** various maths commands ***

\newcommand\X{\times}
\newcommand{\tow}{\rightharpoonup}
\newcommand{\pa}{\partial}
\newcommand\rot{{\rm Rot}}


%  ************  end my definitions  *******************


\begin{document}


\title{Masters in Computer Vision\\Software Engineering Project\\2013-2014}
\author{Ozan\\Oksana\\Klemen\\Natalia\\
{\small
January 2014
}
}
\bigskip



%\begin{abstract}
%This project will investigate some aspects of rational approximation to
%real numbers.
%It is well known that any irrational number can be estimated arbitrarily
%close by rationals, i.e. for any $\ep>0$ and irrational $x \in \R$,
%there exists integers $p$,$q$ such that
%\begin{equation}  \label{abs_basic_app.eq}
%\left| {x-\frac{p}{q}} \right| < \ep
%\end{equation}
%This is known as Diophantine approximation, named after Diophantus of
%Alexandria.
%
%\end{abstract}


\maketitle


 \setcounter{page}{0}


\tableofcontents


\chapter{Introduction}

As a project at Software Engineering class we had to develop an application that to some extent mimics well known Google Earth. The main idea is to give an user who is unfamilliar with Le Creusot the tool that enables him to search for shortest path between two points on the map, enables him to create an itinerary containing different types of points of interest. We had to do all this in C++ and in Matlab.

\chapter{Project management}
In this chapter we want to present how our group organized and managed this project. As we all had a bit of software designing experiences, we knew how important a proper plan and research before the implementation is. Unfortunately, we were also aware that no matter how good the plan is, we will be forced to adjust it during the implementation, because of the things we did not take into account while planning or because some things turned out to be different to what we assumed. Because of that we decided to follow the iterative and incremental development model, which enabled us to adjust our plans after each of the implementation iterations. A schematic representation of the iterative and incremental development model can be seen on figure []. 

\section{Basic building blocks}
To be able to fully manage the project at all times we decided to split the project into four main parts (user interface, map representation, path algorithms and database).  
\section{Softwares used for project management}
Github, skype etc.
\section{Meetings}
some bullshit about that

\chapter{Initial planning}
As already mentioned in the previous chapter, we decided to use the iterative and incremental development model. Before starting the iterations we spend quite a lot of time on the initial planning, with special attention to user and general project requirements analysis. From the project's instructions we were able to identify XXX major user requirements:
\begin{itemize}
\item user should be able to enter point either by:
\begin{itemize}
\item mouse click
\item specifying latitude and longitude
\item selecting a point of interest
\end{itemize}
\item find shortest path from point A to point B (by foot or by car)
\item find all points of interest in a certain radius from point A (by foot or by car)
\item construct an itinerary from point A to point B with points of interest in between (with max distance limit)
\end{itemize}
For each of them we constructed use case diagram to help us while implementing all of them.
\section{Shortest path A -> B}
\section{title}

\chapter{Plan of implementation}
Just how we split matlab and c++ in gui parts

\chapter{Path algorithms}
\section{Street data}
What did we use...open street map...shot introduction into nodes, relations etc.
\section{DB class structure}
\section{Algorithms - shortest path}
\subsection{A*}
\section{Radius search}
\section{Bicycle search}
\chapter{GUI}
\section{C++}
\subsection{UI}
\subsection{OpenGL}
\section{Matlab}
\subsection{UI}
\subsection{library for maps}
\chapter{c++ vs matlab}




\begin{thebibliography}{99}

\bibitem{Bovey}
J. D. Bovey, M. M. Dodson,
The Hausdorff dimension of systems of linear forms
{\em Acta Arithmetica}
(1986) 337-358.

\bibitem{Cassels}
J. W. S. Cassels,
{\em An Introduction to Diophantine Approximation},
Cambridge University Press, 1965.



\end{thebibliography}


\end{document}
